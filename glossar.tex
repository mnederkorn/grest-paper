\usepackage[latin1]{inputenc}
\usepackage[toc]{glossaries}
\makeglossaries
\renewcommand{\glossaryname}{Glossar}

\newglossaryentry{AES}{name=AES, description={Advanced Encryption Standard, auch Rijndael-Algorithmus; symmetrischer Blockchiffre \cite{rijndael}}}

\newglossaryentry{ASCII}{name=ASCII, description={American Standard Code for Information Interchange; Zeichensatz mit 7-Bit-Kodierung}}

\newglossaryentry{Base64}{name=Base64, description={Kodierung, die 8-Bit Bin�rdaten auf eine 6-Bit gro�e Zeichenmenge abbildet}}

\newglossaryentry{Brute-Force-Angriff}{name=Brute-Force-Angriff, description={Sicherheitsangriff auf ein Kryptosystem durch das Ausprobieren aller m�glichen L�sungen. In der Theorie immer erfolgreich, in der Praxis meist aufgrund eines zu gro�en L�sungsraums nicht in einem akzeptablen Zeitraum durchf�hrbar}}

\newglossaryentry{CFB-Modus}{name=CFB-Modus, description={Cipher Feedback Mode; Betriebsmodus f�r Blockchiffre, siehe Kapitel \ref{cfb}}}

\newglossaryentry{Denial-of-Service-Angriff}{name=Denial-of-Service-Angriff, description={Angriff auf die Verf�gbarkeit eines Systems, h�ufig durch �berlastung eines Dienstes}}

\newglossaryentry{Firewall}{name=Firewall, description={technisches Sicherheitskonzept zum Schutz von Rechnernetzen und deren \glslink{Host}{Hosts}}}

\newglossaryentry{Handshake}{name=Handshake, description={Er�ffnung einer Sitzung in einem Kommunikationsprotokoll}}

\newglossaryentry{Hash-Algorithmus}{name={Hash-Algorithmus}, plural={Hash-Algorithmen}, description={bildet Eingabedaten eindeutig auf einen Ausgabewert (Hash-Wert) ab. Der Hash-Wert eines kryptographisch sicheren Algorithmus erm�glicht keine R�ckschl�sse auf die Eingabedaten}}

\newglossaryentry{HMAC}{name=HMAC, description={Keyed-Hash Message Authentication Code, siehe Kapitel \ref{hmac}}}

\newglossaryentry{Host}{name=Host, plural=Hosts, description={Netzkomponente, die dem Rechnernetz Dienste zur Verf�gung stellt}}

\newglossaryentry{HTTP}{name=HTTP, description={Hypertext Transfer Protocol; Daten�bertragungsprotokoll}}

\newglossaryentry{HTTP-Proxy}{name=HTTP-Proxy, description={Netzkomponente, die \gls{HTTP}-Verbindungen vermittelt}}

\newglossaryentry{HTTPS}{name=HTTPS, description={HTTP �ber \gls{SSL/TLS}}}

\newglossaryentry{IP}{name=IP, description={Internet Protocol; Netzwerkprotokoll auf Vermittlungsschicht des \glslink{OSI-Modell}{OSI-Modells}}}

\newglossaryentry{IP-Spoofing}{name=IP-Spoofing, description={Versenden von \gls{IP}-Paketen mit gef�lschter Quelladresse zur Vort�uschung einer fremden Identit�t}}

\newglossaryentry{Java SE}{name=Java SE, description={Java Platform, Standard Edition; Java-Laufzeitumgebung f�r Desktop- und Serversysteme}}

\newglossaryentry{kerckhoff}{name={Kerckhoffs' Prinzip}, description={Kryptographisches Prinzip von Auguste Kerckhoffs (1835--1903), nach dem die Sicherheit eines Kryptosystems durch die Geheimhaltung des Schl�s\-sels und nicht der Geheimhaltung des Algorithmus hergestellt wird}}

\newglossaryentry{Media Type}{name={Media Type}, description={auch MIME Type; Klassifizierungsschema f�r Datenformate \cite{rfc2046}}}

\newglossaryentry{MITM-Angriff}{name=Man-in-the-Middle-Angriff, description={Sicherheitsangriff auf eine Daten�bertragung, bei dem sich ein An\-grei\-fer zwischen den Kommunikationsteilnehmern befindet und deren �bertragung abh�ren und manipulieren kann}}

\newglossaryentry{OSI-Modell}{name=OSI-Modell, description={siebenschichtiges Referenzmodell zum Entwurf von Kommunikationsprotokollen}}

\newglossaryentry{Padding}{name=Padding, description={Auff�llen von Nutzdaten mit inhaltslosen Bytes, um eine Datenstruktur vorgegebener Gr��e einzuhalten}}

\newglossaryentry{PMS}{name=PMS, description={Property Management System; Anwendung zur elektronischen Verwaltung und Abrechnung im Hotelgewerbe}}

\newglossaryentry{Polling}{name=Polling, description={zyklisches Abfragen eines Zustands}}

\newglossaryentry{POSIX}{name=POSIX, description={Portable Operating System Interface; standardisierte Betriebssystemschnittstelle f�r Anwendungen, insbesondere im Unix-Bereich g�ngig}}

\newglossaryentry{Replay-Angriff}{name={Replay-Angriff}, plural={Replay-Angriffe}, description={Sicherheitsangriff durch Wiedereinspielung zuvor aufgezeichneter Daten}}

\newglossaryentry{RFID}{name=RFID, description=Radio Frequency Identification; elektronisches Identifizierungsverfahren}

\newglossaryentry{Router}{name=Router, description={vermittelnde Komponente in einem Rechnernetz}}

\newglossaryentry{RS-232}{name=RS-232, description=eigentlich EIA/TIA-232; serielle Schnittstelle zum Anschluss von Peripherieger�ten}

\newglossaryentry{Seitenkanalangriff}{name=Seitenkanalangriff, plural=Seitenkanalangriffe, description={Sicherheitsangriff auf eine bestimmte Implementation eines kryptographischen Verfahrens durch Beobachtung externer Parameter, etwa der verbrauchten Rechenzeit oder der Speichernutzung}}

\newglossaryentry{SHA}{name=SHA, description={Secure Hash Algorithm; eine Familie von kryptographischen \glslink{Hash-Algorithmus}{Hash-Al\-go\-rith\-men} mit Ausgabegr��en von 160 bis 512 Bit}}

\newglossaryentry{Sniffing}{name=Sniffing, description={Abh�ren des Datenverkehrs im lokalen Netzwerk}}

\newglossaryentry{SSL/TLS}{name={SSL/TLS}, description={Secure Sockets Layer/Transport Layer Security; Kommunikationsprotokoll zur kryptographisch sicheren Daten�bertragung}}

\newglossaryentry{Stateful Packet Inspection}{name={Stateful Packet Inspection}, description={Mechanismus bei \glslink{Firewall}{Firewalls}, um jedes einzelne Datenpaket einer zustandsbehafteten Sitzung zuzuordnen}}

\newglossaryentry{StAX}{name=StAX, description={Streaming API for XML; Programmierschnittstelle zur strombasierten Verarbeitung von XML-Daten}}

\newglossaryentry{TCP}{name=TCP, description={Transmission Control Protocol; Netzwerkprotokoll auf Transportschicht des \glslink{OSI-Modell}{OSI-Modells} \cite{rfc793}}}

\newglossaryentry{USB}{name=USB, description=Universal Serial Bus; serielles Bussystem zum Anschluss von Peripherieger�ten}

\newglossaryentry{UML}{name=UML, description={Unified Modeling Language; grafische Notation zur Systemmodellierung}}

\newglossaryentry{Unixzeit}{name=Unixzeit, description=gel�ufiges Format zur Angabe eines Zeitpunkts: Anzahl Sekunden seit dem 1. Januar 1970 00:00:00 UTC}

\newglossaryentry{URL}{name=URL, description={Uniform Resource Locator; Klassifizierungsschema f�r Ressourcen in einem Rechnernetz \cite{rfc3986}}}

\newglossaryentry{UTF-8}{name=UTF-8, description={8 Bit Unicode Transformation Format; ein Kodierungsformat des Unicode-Zei"-chen"-satzes}}

\newglossaryentry{XMPP}{name=XMPP, description={Extensible Messaging and Presence Protocol; Netzwerkprotokoll, insbesondere f�r Instant Messaging \cite{rfc3920}}}

\newglossaryentry{XML}{name=XML, description={Extensible Markup Language; textbasierte Auszeichnungssprache zur Speicherung und zum Austausch von Daten}}
